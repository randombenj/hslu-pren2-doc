\newpage
\section{Schlussdiskussion}

In diesem Bereich werden die Kosten für die Umsetzung und Entwicklung des Prototypen aufgelistet. Dabei werden die finanziellen Kosten aber auch der zeitliche Aufwand betrachtet. Zudem wird im Bereich "Lessons learned" aufgezeigt, was in diesem Projekt für Erfahrungen mitgenommen werden können. Ein kurzer Ausblick beendet diesen Teilbereich der Dokumentation.

TODO: Fertig stellen

\subsection{Entwicklungskosten}

Die Kosten eines Projektes sind entscheidend. Deshalb wurde dieses Thema bereits früh an fokussiert. Im Pren1 wurde eine Liste mit den erwarteten Preisen erstellt, aufgrund welcher im Pren2 bereits in den ersten Semesterwochen die Teile eingekauft wurden. Mit dieser Vorgehensweise kann zuverlässig die Kostengrenze eingehalten werden.

Im Pren1 wurden bereits zwei Zahnriemen für die Realisierung des ersten Funktionsmusters angeschafft. Die restlichen Kosten sind im Pren2 entstanden und werden nachfolgend aufgeführt.

\begin{center}
\centering
\begin{table}[H]
\begin{tabular}{|l|r|r|r|}
\hline 
\textbf{Bezeichnung} & \textbf{Anzahl} &
\textbf{Stückpreis [CHF]} & \textbf{Gesamtkosten [CHF]} \\
\hline 
Hauptmotor & 1 & 38.95 & 38.95 \\
\hline
Hilfsmotor & 1 & 20.95 & 20.95 \\
\hline
H-Brücke T & 2 & 1.20 & 2.40 \\
\hline
Zahnriemen für Prototyp (Pren1) & 2 & 4.95 & 9.90 \\
\hline
Zahnriemen (Pren2)  & 2 & X & X \\
\hline
Kugellager & 2 & X & X \\
\hline
Taster & 3 & 0.05  & 0.15 \\
\hline
Schrauben div. & 0 & 0 & 0.00 \\
\hline
Fortbewegungsmotoren mit Encoder & 2 & 17.37 & 34.75 \\
\hline
TOF & 2 & 13.25 & 26.5 \\
\hline
Ultraschall & 2 & 0.50 & 1 \\
\hline
Schalter & 1 & 0.10 & 0.10 \\
\hline
LED & 4 & 0.05 & 0.20 \\
\hline
Pi-Kamera V2 & 1 & 29.90 & 29.90 \\
\hline
Kamerakabel & 1 & 5.90 & 5.90 \\
\hline
Nvidia Jetson Nano & 1 & 111.65 & 111.65 \\
\hline
Akku & 1 & 29.95 & 29.95 \\
\hline
Lüfter & 1 & 4.25 & 4.25 \\
\hline
Multiplexer & 1 & 6.60 & 6.60 \\
\hline
Spannungsregler (DC-DC) & 1 & 4.65 & 4.65 \\
\hline
PLA-Filament (3D-Drucker) & 2kg & 24.90 & 24.90 \\
\hline

\hline \hline 
 \textbf{Subtotal} &&& \textbf{CHF XXX.XX}\\
\hline 
\end{tabular}
\caption[Entwicklungskosten]{Entwicklungskosten}
\label{tab:kosten}
\end{table}
\end{center}

\newpage

\subsection{Entwicklungsaufwand}

In den Entwicklungsaufwand wird nicht nur die Arbeit am Prototypen eingerechnet, sondern auch die aufgewendete Zeit für weitere Recherche und Tests ohne Roboter.
Nachfolgend ist diese Auflistung dargestellt.

\begin{center}
\begin{table}[H]
\begin{tabular}{|l|r|r|r|r|}
\hline
\textbf {Was} & \textbf{Aufwand [h]} &
\textbf{Anzahl Personen} & \textbf{Anzahl Wochen} & \textbf{Subtotal [h]}\\
\hline
Pren Block Donnerstag & 3.5 & 6 & 15 & 315 \\
\hline
Pren Block Freitag & 3.5 & 6 & 15 & 315 \\
\hline
Freizeitaufwand (E) & 3.5 & 2 & 15 & 105 \\
\hline
Freizeitaufwand (I) & 3.5 & 2 & 15 & 105 \\
\hline
Freizeitaufwand (M) & 3.5 & 2 & 15 & 105 \\
\hline
Total: & & & & 945 \\ \hline
\end{tabular}
\caption[Entwicklungsaufwand]{Entwicklungsaufwand}
\label{tab:entwicklungsaufwand}
\end{table}
\end{center}

\subsection{Lessons Learned}

Ein Projekt der Grösse von Pren2 ist für das gesamte Team ein Novum. Dabei konnten viele Dinge gelernt werden, welche sich ausschliesslich auf Arbeiten an einem Prototypen beziehen.

So wurde realisiert, dass ein Prototyp bereits früh umgesetzt werden sollte um so schnell wie möglich erste Funktionen zu testen und auf diesem Prototypen aufbauen zu können. Ebenfalls wurde gelernt, das ein solcher erster Prototyp aus 3D-gedrucktem Filament einfach, schnell und flexibel auf Änderungen ist. Dieses Filament hat jedoch auch Nachteile. Spezifisch die Querstabilität einzelner Teile darf nicht vernachlässigt werden und im fertigen Produkt sollten keine gedruckten Teile beinhalten. Trotzdem können auch stark belastete Zahnräder funktionieren, wenn auf das Material und die Voraussetzungen eingegangen wird und die Teile dementsprechen angepasst werden. Ausserdem währe ein selbstgezeichnetes PCB eine Vereinfachung für das Kabelmanagement gewesen. 

Im Bezug zur Gruppenarbeit konnten viele Dinge mitgenommen werden. So ist die Kommunikation zwischen den Teammitgliedern entscheidend. Ausserdem wurde erkannt, dass je weiter der Prototyp fortschreitet die Disziplinen verschwimmen und in gewissen Situationen jeder in allen Disziplinen aushelfen kann. Es war entscheidend, diese Auflösung der Disziplinen aufrecht zu erhalten. So konnte viel Wissen weitergegeben werden.

Das Projektmanagement ist von entscheidender Bedeutung für die realisierung eines solchen Projektes. Auch hier konnten einige Dinge verinerlicht werden. So ist ein gutes Konzept wichtig. Das Konzept aus Pren1 war sehr detailliert, was im Pren2 viele Situationen mit Problemen einfacher gestaltete. Ausserdem sollten keine zu grossen Änderungen inmitten des Projektes durchgeführt werden ohne genügende und saubere Vorbereitung. Der Wechsel des Raspberry Pis auf das Nvidia Jetson hat unnötig viel Zeit gekostet jedoch waren mit diesem Wechsel auch Erfahrungen verknüpft womit ein solche Situation in zukünftigen Projekten nicht mehr eintreten sollte. Die ausergewöhnliche Gegebenheit mit Corona hat den Teammitgliedern viel Flexibilität abverlangt. Das ist ein weiterer Punkt, welcher in einer solchen Arbeit wichtig ist. Es gab insgesamt zwei Ausfälle aufgrund einer Erkrankung oder Quarantäne in welcher die restlichen Teammitglieder flexibel und zuverlässig einspringen konnte.



\subsection{Kritische Würdigung der Arbeit}

Das Projekt ist gut gelaufen. Das Team hat fantastisch Zusammengearbeitet. Die Dokumentation ist detailliert und umfangreich. Der Roboter selbst ist ausgereift und funktioniert. Es gibt jedoch einige Punkte welche besser hätten umgesetzt werden können. So hätte der Antrieb der Fortbewegung übersetzt werden oder mit einem besseren Encoder versehen sollen. Dies auschliesslich um die Genauigkeit der Räder besser bestimmen zu können.


\subsection{Offene Punkte, Risiken, Ausblick}

TODO: Offene Punkte

\newpage

\subsubsection{Ausblick}

Das Projekt wird mit dem Ende des Moduls Pren2 und dem Wettbewerb beendet. Es ist nicht geplant, an diesem Prototypen weiterzuarbeiten.



\subsubsection{Risiken}

Viele Risiken können mithilfe eines konsequenten Risikomanagements drastisch reduziert werden. Im Pren1 konnten bereits einige Risiken für das Modul Pren2 identifiziert werden, welche im Prototypenbau systematisch umgangen oder vermindert werden konnten. Die vollständige, iterativ/inkrementell erstellte, Risikoliste mit den entsprechenden Korrekturmassnahmen ist im Anhang zu finden. Die schwerwiegensten Risiken werden nachfolgend kurz erläutert.

\begin{center}
\begin{table}[H]
    \begin{tabularx}{\textwidth}{|l|X|X|}
        \hline
        \textbf{ID} & \textbf{Titel} & \textbf{Vorbeugende Massnahmen} \\ \hline
        R15 & Anschlag bei Drehbewegung in der Nähe eines Hindernisses bei minimalem Durchgang von 400 mm. & Lösungskonzept erarbeiten oder Dimension so wählen, dass Rotation immer möglich ist. \\ \hline
        R18 & Hindernisse werden nicht erkannt & Früh im PREN2 testen und Lichtverhältnisse miteinkalkulieren. \\ \hline
        R17 & Nicht alle Aspekte bei der Berechnung der Hubbewegung miteinkalkuliert. Hubbewegung funktioniert nicht. & Möglichst früh in \acrshort{pren2}, die Hubbewegung testen.\\ \hline
    \end{tabularx}
    \caption{Top 3 Risiken in Hinsicht auf PREN2}
    \label{tab:risikomanagement-ausblick}
\end{table}
\end{center}










