\newpage
\section{Schlussdiskussion}

In diesem Abschnitt werden die Kosten für die Umsetzung und Entwicklung des Prototypen aufgelistet. Dabei werden die finanziellen Kosten aber auch der zeitliche Aufwand betrachtet. Zudem wird im Bereich \glqq Lessons learned \grqq{} aufgezeigt, was in diesem Projekt für Erfahrungen mitgenommen werden können. Ein kurzer Ausblick beendet diesen Teilbereich der Dokumentation.

\subsection{Entwicklungskosten}

Die Kosten eines Projektes sind entscheidend. Deshalb wurde dieses Thema bereits früh an fokussiert. Im  PREN 1 wurde eine Liste mit den erwarteten Preisen erstellt, aufgrund welcher im PREN 2 bereits in den ersten Semesterwochen die Teile eingekauft wurden. Mit dieser Vorgehensweise kann zuverlässig die Kostengrenze eingehalten werden.

Im PREN 1 wurden bereits zwei Zahnriemen für die Realisierung des ersten Funktionsmusters angeschafft. Die Gesamtkosten werden anschliessend aufgeführt.

\begin{center}
\centering
\begin{table}[H]
\begin{tabularx}{\textwidth}{|X|l|X|X|}
\hline 
\textbf{Bezeichnung} & \textbf{Anzahl} &
\textbf{Stückpreis [CHF]} & \textbf{Gesamtkosten [CHF]} \\
\hline 
Hauptmotor & 1 & 38.95 & 38.95 \\
\hline
Hilfsmotor & 1 & 20.95 & 20.95 \\
\hline
H-Brücke T & 2 & 1.20 & 2.40 \\
\hline
Zahnriemen für Prototyp & 1 & 4.95 & 4.95 \\
\hline
Zahnriemen (Pren2)  & 2 & 20.48 & 40.96 \\
\hline
Kugellager & 2 & 8.78 & 17.56 \\
\hline
Kugellager 2 & 2 & 7.40 & 14.80  \\
\hline
Taster & 3 & 0.05  & 0.15 \\
\hline
Fortbewegungsmotoren & 2 & 17.37 & 34.75 \\
\hline
Radnaben & 2 & 3.50 & 7.00 \\
\hline
Ultraschall & 3 & 1.50 & 4.50 \\
\hline
Schalter & 1 & 0.10 & 0.10 \\
\hline
LED & 4 & 0.05 & 0.20 \\
\hline
Pi-Kamera V2 & 1 & 27.75 & 27.75 \\
\hline
Kamerakabel & 1 & 5.50 & 5.50 \\
\hline
Nvidia Jetson Nano & 1 & 111.65 & 111.65 \\
\hline
Akku & 2 & 29.95 & 59.90\\
\hline
Lüfter & 1 & 4.25 & 4.25 \\
\hline
Multiplexer & 2 & 6.60 & 13.20 \\
\hline
Spannungswandler & 1 & 4.65 & 4.65 \\
\hline
PWM-Extension Board & 1 & 11.90 & 11.90 \\
\hline
Jumper Kabel & 1 & 6.50 & 6.50 \\
\hline
microSDXC 128GB & 1 & 26.70 & 26.70 \\
\hline
PLA-Filament & 1kg & 24.90 & 24.90 \\
\hline

\hline \hline 
 \textbf{Subtotal} &&& \textbf{CHF 484.12}\\
\hline 
\end{tabularx}
\caption[Entwicklungskosten]{Entwicklungskosten}
\label{tab:kosten}
\end{table}
\end{center}

\newpage

\subsection{Entwicklungsaufwand}

In den Entwicklungsaufwand wird nicht nur die Arbeit am Prototypen eingerechnet, sondern auch die aufgewendete Zeit für weitere Recherche und Tests ohne Roboter.
Die nachfolgende Auflistung bezieht sich nur auf die 15 Semesterwochen. Die zusätzliche Arbeitszeit zwischen Semester und Wettbewerb wird nicht eingerechnet.

\begin{center}
\begin{table}[H]
\begin{tabularx}{\textwidth}{|X|X|X|X|X|}
\hline
\textbf {Was} & \textbf{Aufwand [h]} &
\textbf{Anzahl Personen} & \textbf{Anzahl Wochen} & \textbf{Subtotal [h]}\\
\hline
Pren Block Donnerstag & 3.5 & 6 & 15 & 315 \\
\hline
Pren Block Freitag & 4 & 6 & 15 & 360 \\
\hline
Pren Block Samstag & 3.5 & 4 & 8 & 112 \\
\hline
Freizeitaufwand (E) & 4 & 2 & 15 & 120 \\
\hline
Freizeitaufwand (I) & 5.5  & 2 & 15 & 165 \\
\hline
Freizeitaufwand (M) & 5 & 2 & 15 & 150 \\
\hline
Total: & & & & 1222 \\ \hline
\end{tabularx}
\caption[Entwicklungsaufwand]{Entwicklungsaufwand}
\label{tab:entwicklungsaufwand}
\end{table}
\end{center}

\newpage

\subsection{Lessons Learned}


Ein Projekt der Grösse von Pren2 ist für das gesamte Team ein Novum. Dabei konnten viele Dinge gelernt werden, welche sich ausschliesslich auf Arbeiten an einem Prototypen beziehen.

So wurde realisiert, dass ein Prototyp bereits früh umgesetzt werden sollte, um so schnell wie möglich erste Funktionen zu testen und auf diesem Prototypen aufbauen zu können. Ebenfalls wurde gelernt, das ein solcher erster Prototyp aus 3D-gedrucktem Filament einfach, schnell und flexibel auf Änderungen ist. Ausserdem wäre ein selbst erstelltes PCB eine Vereinfachung für das Kabelmanagement gewesen, womit einige Probleme mit Wackelkontakten hätten verhindert werden können.
Des weiteren wäre für die Kontrolle und die zuverlässige Ansteuerung der Sensoren ein Microcontroller von Vorteil gewesen. Damit wäre eine zeitliche Steuerung und zeitliche Abmessung vereinfacht und genauer durchführbar gewesen. Das Einkaufen der Komponente sollte nicht dem Zufall überlassen werden. So ist es wichtig, sicherzustellen, dass die Teile tatsächlich bestellt und nicht vergessen wurden. Eine weitere Lektion die gelernt wurde ist, dass darauf geachtet werden sollte, dass die Encoder eine genügende Auflösung bieten. Und wenn eine genügende Auflösung nicht erreicht werden kann, sollte eine bessere Auflösung durch genauere Encoder oder eine Übersetzung erreicht werden. Falls dies nicht möglich ist, sollte eine  Alternative beispielsweise mit einer Kamera oder Distanzsensoren eingebunden werden, mit welcher die Drehung überprüft und kontrolliert werden kann. Dies um sicherzustellen, dass eine 90\textdegree\ Drehung genau und reproduzierbar erreicht werden kann.

Im Bezug zur Gruppenarbeit konnten viele Dinge mitgenommen werden. So ist die Kommunikation zwischen den Teammitgliedern entscheidend. Ausserdem wurde erkannt, dass je weiter der Prototyp fortschreitet die Disziplinen verschwimmen und in gewissen Situationen jeder in allen Disziplinen aushelfen kann und sollte. Es war entscheidend, diese Auflösung der Disziplinen aufrecht zu erhalten. So konnte viel Wissen weitergegeben werden und das meiste aus dem Prototypen herausgeholt werden.

Das Projektmanagement ist von entscheidender Bedeutung für die Realisierung eines solchen Projektes. Auch hier konnten einige Dinge verinnerlicht werden. So ist ein gutes Konzept wichtig. Das Konzept aus \acrshort{pren1} war sehr detailliert, was im \acrshort{pren2} viele Situationen mit Problemen einfacher gestaltete. Ein weiterer Punkt ist, dass bezüglich der Sprintplanung die Software Gitlab hätte verwendet werden können. Dadurch ergibt sich die Möglichkeit, einzelne Issues für die Sprints aufzulisten, abzuarbeiten und die aufgewendete Zeit einzutragen. Zum Schluss können die Burn-Down-Charts der Sprint ausgegeben werden und in der Dokumentation beschrieben werden. Ausserdem sollten keine zu grossen Änderungen inmitten des Projektes durchgeführt werden ohne genügende und saubere Vorbereitung. Der Wechsel des Raspberry Pi's auf das Nvidia Jetson Nano hat unnötig viel Zeit gekostet, jedoch waren mit diesem Wechsel auch Erfahrungen verknüpft, womit ein solche Situation in zukünftigen Projekten nicht mehr eintreten sollte.
Die aussergewöhnliche Gegebenheit mit Corona hat den Teammitgliedern viel Flexibilität abverlangt. Das ist ein weiterer Punkt, welcher in einer solchen Arbeit wichtig ist. Es gab insgesamt zwei Ausfälle aufgrund einer Erkrankung oder Quarantäne in welcher die restlichen Teammitglieder flexibel und zuverlässig einspringen konnte.

\newpage

\subsection{Kritische Würdigung der Arbeit}

Das Projekt ist gut gelaufen. Das Team hat gut zusammengearbeitet. Die einzelnen Module des Roboters funktionieren. Die Dokumentation ist detailliert und umfangreich. Es gibt jedoch einige Punkte, welche besser hätten umgesetzt werden können. So hätte der Antrieb der Fortbewegung übersetzt werden oder mit einem besseren Encoder versehen werden sollen. Dies ausschliesslich um die Genauigkeit der Räder besser bestimmen zu können. Ein Encoder für die Hubbewegung hätte auch einige Probleme vereinfacht.

Das Fortbewegungskonzept wurde aufgrund der Einfachheit gewählt. Ein alternatives Konzept, bei welchem sich der Roboter seitlich verschieben kann ohne sich abzudrehen, wäre im Nachhinein unter Umständen ein einfacherer Ansatz gewesen.


\subsection{Offene Punkte}

Es gäbe verschiedene Möglichkeiten, den Roboter zu verbessern, für welche jedoch die Zeit fehlt. So wäre es interessant gewesen, zu sehen, wie der Roboter sich mit besseren TOF oder Ultraschallsensoren verhaltet. 

Des weiter ist die Positionsbestimmung vor der Treppe mit den Distanzsensoren und Trigonometrie nicht besonders ausgereift. Hierfür wäre allenfalls ein Modul, welches die Position vor der Treppe anhand des Geländers und der Stufen mithilfe der Kamera und Bildverarbeitung berechnet, besser geeignet beziehungsweise genauer.


\subsection{Ausblick}

Das Projekt wird mit dem Ende des Moduls \acrshort{pren2} und dem Wettbewerb beendet. Es ist nicht geplant, an diesem Prototypen weiterzuarbeiten. Der Roboter wird nach Abschluss der Hochschule Luzern übergeben.

Ein weiterführbarkeit des Projektes wäre jedoch gegeben. Neben den bereits verbauten Motoren ist sowohl der Platz, eine freie H-Brücke wie auch mehrere PWM Signalgeber für Upgrades vorhanden. Die Akkukapazität würde etwaige Erweiterung ebenfalls unterstützen. Das vorhandene 5 Volt Niveau wie auch mehrere freie Pins am Jetson könnten gegebenenfalls für weitere Sensoren oder allgemeine Komponente verwendet werden.

\newpage

\subsection{Risiken}

Viele Risiken können mithilfe eines konsequenten Risikomanagements drastisch reduziert werden. In diesem Abschnitt werden die grössten Risiken bezüglich des Wettbewerbes aufgelistet.

\begin{center}
\begin{table}[H]
    \begin{tabularx}{\textwidth}{|X|X|}
        \hline
        \textbf{Titel} & \textbf{Vorbeugende Massnahmen} \\ \hline
        Genaues Ausrichten vor der Treppe nicht möglich & TOF-Sensoren mit Ultraschall ausgetauscht. Code ausgiebig überarbeitet. \\ \hline
        Rotation auf der Treppe ungenau & Vielzahl an Tests und weiterausarbeitung des Codes. Die Drehung mithilfe der Sensoren überprüfen und nachbessern.
        \\ \hline
       Guten Punkte des Roboters können nicht gezeigt werden & Die ersten Schritte des Ablaufes werden weiter verfeinert, um sicherzustellen, dass der Roboter seine Stärke, das Treppensteigen, ausführlich demsonstrieren kann. 
       \\ \hline
    \end{tabularx}
    \caption{Top 3 Risiken in Hinsicht auf Wettbewerb}
    \label{tab: }
\end{table}
\end{center}
