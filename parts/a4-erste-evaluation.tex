\section{Erste Evaluation}
Für den ersten Evaluationsschritt werden Killerkriterien formuliert, welche aus der Anforderungsliste und der Aufgabestellung abgeleitet werden. Es soll eine Erstevaluation durchgeführt werden, um die vielen recherchierten Lösungsmöglichkeiten einzugrenzen. Die verbleibenden, möglichen Teillösungen werden anschliessend in einem morphologischen Kasten zusammengetragen. Mit dem morphologischen Kasten werden die Teillösungen zu mehreren Gesamtlösungen kombiniert. Diese Gesamtlösungen werden dann in einem zweiten Evaluationsschritt mit einer Nutzwertanalyse bewertet.

In einem ersten Schritt werden Killerkriterien erstellt, auf welche sich die nachfolgenden Evaluationen beziehen:

\textbf{Killerkriterien}
\begin{enumerate} 
	\item Gerät funktioniert bei jeder Witterung. Gerät darf nicht von Wind und Regen beeinflusst werden.
	\item Das ganze Gerät muss sich auf dem Zielpodest (Podest am oberen Ende der Treppe) befinden.
	\item Die Treppenstufen sind aus Holz oder Aluminium (nicht magnetisch).
	\item Die Hindernisse liegen zufällig flach oder hochkant verteilt auf der Treppe.
	\item Aufgrund des vierten Killerkriteriums muss auf einem Treppentritt nach rechts oder links ausgewichen werden können.
    \item Budget von CHF 500 nicht überschreiten
	\item Treppe und Treppenstufen finden/erkennen
	\item Hindernisse finden/erkennen
	\item Das Fahrzeug muss fähig sein zwei Läufe zu absolvieren.
	\item Das gesamte Gewicht von 3kg nicht überschreiten
	\item Das Gerät soll mit einem Not-Aus Taster sofort gestoppt werden können. 
	\item Die Aufträge welche quittiert werden müssen, sind im vorhinein bekannt.
	\item Das Gerät muss die Aufgabe autonom bewältigen. Es ist nicht erlaubt, das Gerät von einem externen stationären Rechner aus oder fremd zu steuern.
	\item Das Gerät muss seine Position auf mehrere cm genau bestimmen können (Abstand zwischen zwei Hindernissen beträgt mindestens 400mm).
\end{enumerate}



\subsection*{Fortbewegung und Treppensteigen}
\textbf{Ausgeschiedene Teilfunktionen}
In der Tabelle \ref{tab:ausgeschiedene-teilfunktionen-fortbewegung} werden die ausgeschiedenen Teilfunktionen in der Fortbewegung, Lenkung und im Treppensteigen rot markiert. In der nächsten Tabelle \ref{tab:ausgeschiedene-teilfunktionen-hindernisse}, werden die ausgeschiedenen Lösungsmöglichkeiten, um Hindernisse zu überwinden rot eingefärbt. 
\begin{center}
\begin{table}[ht!]
    \begin{tabular}{l|l|l}
        \textbf{Fortbewegung} & \textbf{Treppensteigen} & \textbf{Lenkung} \\
        Normale Räder & Spezielle Treppenräder & Lenkachse \\
        Omnidrive–Allseitenräder & Hebemechanismus: 3-Teilig & Knicklenkung \\ 
        Omnidrive–Mecanumräder & Hebemechanismus: Raufkappen & Panzerlenkung \\
        Omnidrive–Fahrdrehmodul & Hebemechanismus: Aufstapeln und Ausfahren & Roomba-Prinzip \\
        \cellcolor{red}Propeller,Rotoren & \cellcolor{red}Katapult & \\
        Beine & Sprungfeder & \\
         & \cellcolor{red}Bahn über Treppe ausfahren für kleines Fahrzeug & \\ 
         & \cellcolor{red}Schlange mit Magneten & \\
    \end{tabular}
    \caption{Teilfunktionen Mechanik 1}
    \label{tab:ausgeschiedene-teilfunktionen-fortbewegung}
\end{table}\end{center}

\begin{center}
\begin{table}[h!]
    \begin{tabular}{l}
        \textbf{Hindernisse überwinden}\\
        Umgehen\\
        \cellcolor{red}Überfahren\\
        \cellcolor{red}Überfliegen\\
        \cellcolor{red}Besteigen\\
    \end{tabular}
    \caption{Teilfunktionen Mechanik 2}
    \label{tab:ausgeschiedene-teilfunktionen-hindernisse}
\end{table}
\end{center}

Teilfunktion Fortbewegung:
\begin{itemize}
    \item Propeller und Rotoren scheiden aufgrund des ersten Killerkriteriums aus. Ein Flug ist wegen der ungewissen Witterung problematisch. Sollte es am Wettkampftag regnen und winden, so wird es schwierig für eine Drohne auf Kurs zu bleiben und sich auf dem Zielpodest richtig beim Zielpiktogramm zu platzieren. Aus diesem Grund wird diese Teillösung nicht weiter in Betracht gezogen.
 \end{itemize}

Teilfunktion Treppensteigen:
\begin{itemize}
    \item Die speziellen Treppenräder fallen wegen des fünften Killerkriteriums raus. Da den Hindernissen ausgewichen werden muss, ist eine Verschiebung auf einem Treppentritt nötig. Dies ist mit einem Fahrzeug mit speziellen Treppenrädern nicht möglich.
    \item Die Teillösung, ein kleines Fahrzeug mit Hilfe eines Katapults über die Treppe auf das Zielpodest zu schiessen, kann aufgrund des zweiten Killerkriteriums nicht weiterverfolgt werden. Das ganze Gerät muss sich auf dem Zielpodest befinden. Dies ist mit einem Katapult nicht möglich, da sich zwar das katapultierte kleine Fahrzeug im Ziel befindet, jedoch das Katapult an Ort und Stelle im Startbereich verbleibt.
    \item Auch die Idee, eine Bahn von einem grossen Fahrzeug über die Treppe auszufahren um mit einem kleinen Fahrzeug diese Bahn zu befahren, scheitert am zweiten Killerkriterium. Auch hier wäre zwar das kleine Fahrzeug auf dem Zielpodest, aber das grosse Bahnleger-Fahrzeug würde im Startbereich verbleiben.
    \item Die Teillösung einer Schlange mit Magneten an den Enden, um mit Hilfe dieser Magneten an den Vorderseiten der Treppenstufen die Treppe heraufzuklettern fällt weg. Dies aufgrund des dritten Killerkriteriums. Da die Treppenstufen aus Holz oder Aluminium sein werden, wird der Einsatz von Magneten nicht möglich sein, da diese Materialien nicht magnetisch sind.
\end{itemize}

Teilfunktion Hindernisse überwinden:
\begin{itemize}
    \item Die Teillösungen die Hindernisse zu überfahren oder zu besteigen sind nicht möglich wegen dem vierten Killerkriteriums. Da die Hindernisse auch hochkant platziert werden können, können diese nicht einfach überfahren oder bestiegen werden.
    \item Die Teillösung die Hindernisse zu überfliegen fällt weg, weil die Rotoren und Propeller bereits ausgeschieden sind.
 \end{itemize}

\textbf{Verbleibende Teillösungen für die zweite Evaluation}
Die Tabelle \ref{tab:verbleibende-teillösungen-mechanik-1} und die Tabelle \ref{tab:verbleibende-teillösungen-mechanik-2} zeigen die noch verbleibenden Teillösungen für die zweite Evaluation.
\begin{center}
\begin{table}[h!]
    \begin{tabular}{l|l|l}
        \textbf{Fortbewegung} & \textbf{Treppensteigen} & \textbf{Lenkung}\\
        Normale Räder & Spezielle Treppenräder & Lenkachse\\ 
        Omnidrive – Allseitenräder & Hebemechanismus: 3-Teilig & Knicklenkung\\ 
        Omnidrive – Mecanumräder & Hebemechanismus: Raufkappen & Panzerlenkung\\ 
        Omnidrive – Fahrdrehmodul & Hebemechanismus: Aufstapeln und Ausfahren & Roomba - Prinzip\\ 
        Beine & Sprungfeder &\\
    \end{tabular}
    \caption{Verbleibende Lösungen Mechanik 1 nach der ersten Evaluation}
    \label{tab:verbleibende-teillösungen-mechanik-1}
\end{table}
\end{center}

\begin{center}
\begin{table}[h!]
    \begin{tabular}{l}
        \textbf{Hindernisse überwinden}\\
        Umgehen
    \end{tabular}
    \caption{Verbleibende Lösungen Mechanik 2 nach der ersten Evaluation}
    \label{tab:verbleibende-teillösungen-mechanik-2}
\end{table}
\end{center}
\newpage


\subsection*{Sensoren und Stromversorgung}
\textbf{Ausgeschiedene Lösungsmöglichkeiten}

Orientierung
\begin{itemize}
    \item GPS hat eine Genauigkeit von ca. einem Meter und ist für eine präzise Positionierung auf der Treppe oder auf dem Startfeld aufgrund des Killerkriterium 14 ungeeignet.
\end{itemize}
Antrieb
\begin{itemize}
    \item Ein Verbrennungsmotor aufgrund des hohen Gewichts nicht geeignet (Killerkriterium 10). Ausserdem muss die Mechanische Energie zuerst in elektrische Energie für die elektrischen Bauteile umgewandelt werden.
    \item Ein Federwerk ist wegen der geringen Leistung nicht geeignet. Nur durch einmaliges Aufziehen des Federwerks könnte der Roboter keine zwei Läufe absolvieren (Killerkriterium 9).
\end{itemize}
Not-Aus
\begin{itemize}
    \item Das Killerkriterium 11 verlangt einen Not-Aus Taster. Somit fallen die Funktionen 'Negative Beschleungiung' und 'Energie kappen' weg.
\end{itemize}
Energiequelle
\begin{itemize}
    \item Brennstoffzelle: Um eine genügend hohe Spannung und Kapazität für die elektrischen Bauteile, wie Prozessoren und Antriebsmotoren, zu erreichen, wäre eine hohe Anzahl an Zellen nötig. Dies würde das Gewicht des Roboters stark erhöhen. Mit weniger Zellen ist die Kapazität zu gering und es wäre gemäss Killerkriterium 9 nicht möglich zwei Läufe zu absolvieren.
    \item Photovoltaik: Gemäss dem Killerkriterium 1 muss der Roboter bei jeden Witterungsbedingnugen einsatzbereit sein. Durch bewölkten Himmel würde ein Photovolatikzelle nicht genügend Energie für den Betrieb des Roboters liefern.
\end{itemize}

\textbf{Verbleibende Lösungen für die zweite Evaluation}
Die Tabelle \ref{tab:verbleibende-teillösungen-elektrotechnik} zeigen die noch verbleibenden Teillösungen im Bereich Sensoren und Stromversorgung für die zweite Evaluation.
\begin{center}
\begin{table}[h!]
    \begin{tabular}{l|l|l|l}
        \textbf{Orientierung} & \textbf{Antrieb} & \textbf{Not-Aus} & \textbf{Energiequelle} \\ 
        \hline
        Tastsensor & Bürstenloser DC-Motor & Not-Aus-Schalter & Akku\\  
        Distanzsensor & Linearantrieb &  &\\
        Kamera & Schrittmotor & &\\
        TOF/Lidar & Servomotor & &\\
        Beschleunigungssensor & Getriebemotor & & 
    \end{tabular}
    \caption{Verbleibende Lösungen Elektrotechnik nach der ersten Evaluation}
    \label{tab:verbleibende-teillösungen-elektrotechnik}
\end{table}
\end{center}


\newpage
\subsection*{Auftragsquittierung}

\textbf{Ausgeschiedene Lösungsmöglichkeiten} 

OLED-Display:
\begin{itemize}
    \item Die Lösungsmöglichkeit, den Auftrag mittels eines OLED-Displays dem Publikum zu quittieren kann aufgrund des sechsten Killerkriteriums nicht weiter verfolgt werden. Da die Kosten knapp sind, die Auftragsquittierung nicht zu den Knackpunkten dieses Projekts gehört und das OLED-Display verhältnismässig teuer ist, sollte man hier auf günstigere Möglichkeiten ausweichen.
\end{itemize}

Word Vorlesefunktion + Lautsprecher:
\begin{itemize}
    \item Die Lösungsmöglichkeit den Auftrag mittels der Word Vorlesefunktion auszugeben scheidet aufgrund des 13. Killerkriteriums aus. Ein Vorteil dieser Variante ist, dass man grundsätzlich beliebige Sätze leicht erstellen und ausgeben kann. Da in diesem Projekt jedoch bekannt ist, was quittiert werden muss, fällt dieser Vorteil nicht mehr ins Gewicht.
\end{itemize}

\textbf{Verbleibende Lösungen für die zweite Evaluation}
\begin{itemize}
    \item LCD-Display
    \item Lautsprecher
    \item Audioplayer + Lautsprecher
    \item LED's
\end{itemize}

\newpage

\subsection*{Umgebungserkennung}
\textbf{Ausgeschiedene Lösungsmöglichkeiten}

Google Vision API:
\begin{itemize}
    \item Die Google Vision API kann aufgrund des 13. Killerkriteriums nicht weiter verfolgt werden. Da es sich bei der Google Vision API um ein fertiges Produkt handelt und man über ein REST API auf die Algorithmen für die Bilderkennung und das maschinelle Lernen zugreift.
\end{itemize}

CognitiveJ - Image Analysis mit Java:
\begin{itemize}
    \item Auch diese Lösung fällt aufgrund des 13. Killerkriteriums aus dem Rennen. Hier wird ebenfalls übers Internet auf vorgefertigte Lösungen zugegriffen, welche sich nicht auf dem Gerät selber befinden. Somit würde das Gerät indirekt über einem externen Server gesteuert werden.
\end{itemize}

\textbf{Verbleibende Lösungen für die zweite Evaluation}
\begin{itemize}
    \item TensorFlow 
    \item OpenCV
    \item PyTorch
    \item Scikit Learn
    \item Scikit Image
\end{itemize}

