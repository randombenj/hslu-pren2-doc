\newpage

\section{Bedienungsanleitung und Inbetriebnahme}

Um den Roboter in Betrieb zu nehmen, sind einige wenige Schritte durchzuführen. Diese werden nachfolgend beschrieben.\\

\begin{enumerate}
    \item Lösen Sie die Schrauben der Abdeckhaube und ziehen Sie diese vorsichtig nach oben weg.
    \item Der Akku, welcher sich in der Front des Roboters befindet sollte vor Gebrauch vollständig geladen werden. Lösen Sie den Klettverschluss und ziehen Sie den Akku auf der linken Seite des Gerätes heraus. Laden Sie den Akku ordnungsgemäss.
    \item Setzen Sie den vollen Akku wieder in den Roboter ein. Stellen Sie sicher, dass der Akku ordnungsgemäss hält. Stecken Sie nun den Stecker des Akkus in das Gegenstück des Roboters, welches sich auf dem linken Radkasten befindet. Auf der blau leuchtenden 7-Segment Anzeige oberhalb des Hubbewegungsmotors sollte nun die Spannung des Akkus zu sehen sein. Ist dieser Wert nicht über 16.0 (Volt) wurde der Akku nicht vollständig geladen.
    \item Schalten Sie, nachdem der Akku eingesteckt wurde, den Lautsprecher ein.
    \item Stülpen Sie die Haube des Roboters nun wieder vorsichtig über den Roboter. Achten Sie dabei auf die Orientierung der Haube und darauf, dass die Haube gleichmässig gesenkt wird. Befestigen Sie mit den dazu vorgesehenen Schrauben die Haube an der Grundplatte.
    \item Positionieren Sie den Roboter im Startfeld des Startbereiches und richten Sie diesen auf die Treppe aus.
    \item Kippen Sie den Schalter auf der Oberseite des Roboters auf \glqq ON\grqq{}.
    \item Drücken Sie nun den Startknopf und entfernen Sie sich aus dem Startbereich. Das Gerät wird nun die Startposition überprüfen und danach die Aufgabe angehen. Sobald die Aufgabe beendet ist, wird dies durch die LED's und eine Audioausgabe signalisiert.
\end{enumerate}

\newpage

Wichtige Hinweise:
\begin{itemize}
    \item Stellen Sie sicher, dass die blaue 7-Segment Anzeige stets über dem Wert 12.0 (Volt) ist. Ein unterschreiten dieses Wertes kann den Akku dauerhaft beschädigen oder zersören. Es wird empfohlen, den Akku zu laden, soblad die Anzeige einen Wert kleiner als 14.0 (Volt) anzeigt.
    \item Sollte während einem Lauf ein Fehler auftreten, können Sie den Schalter auf der Oberseite die Motoren stoppen. Der Controller wird davon nicht beeinflusst was ein gezwungenes Ausschalten verhindert.
    \item Achten Sie darauf, dass Sie das Jetson Nano nicht einfach vom Strom nehmen oder den Akku vom laufenden Gerät entfernen. Das Jetson Nano sollte immer ordnungsgemäss heruntergefahren werden. Beispielsweise \texttt{sudo poweroff}.
    \item Der Roboter sollte nicht an den Auslegern aufgehoben werden, da ansonsten das Ritzel des Hubmotors unnötig belastet wird. Besser eignet sich die Grundplatte.
\end{itemize}










