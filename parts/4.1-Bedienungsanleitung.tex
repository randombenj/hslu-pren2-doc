\newpage

\section{Bedienungsanleitung und Inbetriebnahme}

Um den Roboter in Betrieb zu nehmen, sind einige wenige Schritte durchzuführen. Diese werden nachfolgend beschrieben.

1. Lösen Sie die Schrauben der Abdeckhaube und lösen ziehen Sie diese vorsichtig nach oben weg.

2. Der Akku, welcher sich in der Front des Roboters befindet sollte vor Gebrauch vollständig geladen werden. Lösen Sie den Klettverschluss und ziehen Sie deshalb den Akku auf der linken Seite des Gerätes heraus. Laden Sie den Akku ordnungsgemäss.

3. Setzen Sie den vollen Akku wieder in den Roboter ein. Stellen Sie sicher, dass der Akku Ordnungsgemäss hält. Stecken Sie nun den Stecker des Akkus in das Gegenstück des Roboters, welches sich auf dem linken Radlauf des Hubers befindet. Auf der blaue leuchtenden 7-Segment Anzeige oberhalb des Hubbewegungsmotors sollte nun die Spannung des Akkus zu sehen sein. Ist dieser Wert nicht über 16.0 wurde der Akku nicht vollständig geladen.

4. Stülpen Sie die Haube des Roboters nun wieder vorsichtig über den Roboter. Achten Sie sich dabei auf die Orientierung der Haube und darauf, dass die Haube gleichmässig gesenkt wird.
Befestigen Sie mit den dazu vorgesehenen Schrauben die Haube an der Grundplatte.

5. Positionieren Sie den Roboter im Startfeld des Startbereiches und richten Sie diesen auf die Treppe aus.

6. Kippen Sie den Schalter auf der Oberseite des Roboters auf "on".

7. Drücken Sie nun den Startknopf und entfernen Sie sich aus dem Startbereich. Hubert wird nun die Startposition überprüfen und danach die Aufgabe angehen. Sobald die Aufgabe beenden wurde, wird dies durch die LED und den Siegestanz signalisiert.

Wichtige Hinweise:

Stellen Sie sicher, dass die blaue 7-Segment Anzeige stets über dem Wert 12.0 ist. Es wird empfohlen, den Akku zu laden, soblad die Anzeige einen Wert kleiner als 14.0 anzeigt.

Sollte während einem Lauf ein Fehler auftreten, können Sie den Schalter auf der Oberseite die Motoren stoppen.






