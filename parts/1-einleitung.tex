\section{Einleitung}
In der vorliegenden Dokumentation wird im Rahmen des Moduls \acrfull{pren1} ein autonomer Roboter geplant. Dieser Roboter muss ein Bild scannen und anschliessend eine Treppe mit Hindernissen völlig autonom erklimmen. Auf der oberen Plattform angekommen, muss er das Referenzbild finden und dies signalisieren.
Um einen autonomen Roboter zu bauen, braucht es viele verschiedenen Kompetenzen, weshalb die Arbeit in einem interdisziplinären Team bestehend aus Informatik-, Maschinentechnik- und Elektrotechnikstudierenden durchgeführt wird, welche ihr Wissen in das Projekt gewinnbringend einbinden können.

In diesem Bericht wird das Konzept erläutert, wie der Roboter diese Aufgabe zuverlässig erfüllen soll. Von anfänglichen Überlegungen zu einem detaillierten Lösungsansatz. Dabei wird die gestellte Aufgabe in verschiedene Teilbereiche unterteilt. Um für jeden dieser Teilbereiche jeweils die geeignetste Lösung zu finden wird ausgiebig recherchiert und evaluiert. Anschliessend werden diese Teillösungen mit Funktionsmustern überprüft.
Die Punkte, auf welche besonders geachtet werden sind dabei die Einfachheit und Zuverlässigkeit.
Die somit bestimmten Teillösungen werden anschliessend in den finalen Lösungsansatz zusammengefügt. 

Diese Arbeit dient als Vorbereitung für das Modul  \acrfull{pren2}, bei welchem diese Pläne in die Tat umgesetzt werden und das Team den Roboter baut, testet und verbessert. Als Abschluss dieser Module findet am Ende ein Wettbewerb der Hochschule Luzern statt, an welchem alle Teams teilnehmen.

