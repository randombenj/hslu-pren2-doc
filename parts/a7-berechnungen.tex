\section{Berechnungen}

\textbf{Massen und Schwerpunkte}\\

\textbf{Grundkörper}

BILD MACHEN: fertiger Grundkörper auf Waage und Schwerpunkt ermitteln

\begin{figure}[H]
  \includegraphics[width=0.8
  \textwidth]{img/}
  \centering
  \caption{Gewichtsmessung Grundkörper}
\end{figure}

\newpage

\textbf{Ausleger}

BILD MACHEN: fertiger Ausleger auf Waage und Schwerpunkt ermitteln


\begin{figure}[H]
  \includegraphics[width=0.8
  \textwidth]{img/}
  \centering
  \caption{Gewichtsmessung Ausleger}
\end{figure}

-> Gewicht ganzer Roboter dokumentieren
-> Standsicherheitsrechung mit gemessenen Werten bei 2. Hub

\newpage

\textbf{Durchmesser der Wellen}

Auslegung der Wellendurchmesser auf nur TORSION mit Werten aus PREN1








Die Momente, die gebraucht werden, konnten mit den ersten Abmassen, Massen und Schwerpunkten der Komponenten berechnet werden und dienen als Grundlage zur Auswahl der Motoren.

\textbf{Momente bei den Hubbewegungen:} (G:Grundkörper, S:Standfuss, L:Verbindungsleiste)

\textbf{1. Hubbewegung:}

Drehachse 1: Moment zum horizontalen Halten des Grundkörpers
\begin{align*}
    M_{Drehachse 1} &= F_{GG} * 0.0552\ m \\
    &= m_{G} * g * 0.0552\ m \\
    &= 2.9\ kg * 9.81\ m/s^2\ * 0.0552\ m \\
    &= \underline{\underline{1.6\ Nm}}
\end{align*}

Drehachse 2: grösstes Moment am Anfang, in Ausgangsstellung
\begin{align*}
    M_{Drehachse 2} &= F_{G2L} * 0.115\ m + F_{GG} * 0.1748\ m \\
    &= g * (m_{2L} * 0.115\ m + m_{G} * 0.1748)\ m \\
    &= 9.81\ m/s^2\ * (0.38\ kg * 0.115\ m + 2.9\ kg * 0.1748\ m) \\
    &= \underline{\underline{5.4\ Nm}}
\end{align*}

\textbf{2. Hubbewegung:}

Drehachse 1: grösstes Moment, wenn alle Teile horizontal
\begin{align*}
    M_{Drehachse 1} &= F_{G2S} * 0.2852\ m + F_{G2L} * 0.115\ m \\
    &= g * (m_{2S} * 0.2852\ m + m_{2L} * 0.115\ m) \\
    &= 9.81\ m/s^2 * (1.02\ kg * 0.2852\ m + 0.38\ kg * 0.115\ m) \\
    &= \underline{\underline{3.3\ Nm}}
\end{align*}

Drehachse 2: grösstes Moment, wenn Standfüsse horizontal
\begin{align*}
    M_{Drehachse 2} &= F_{G2S} * 0.0552\ m \\
    &= m_{2S} * g * 0.0552\ m \\
    &= 1.02\ kg * 9.81\ m/s^2 * 0.0552\ m \\
    &=\underline{\underline{0.6\ Nm}}
\end{align*}

Motor 1 liefert das Moment, dass in der 2. Drehachse wirkt und Motor 2 liefert das Moment, dass in der 1. Drehachse wirkt.\\
\\
M$_{erforderlich}$ Drehachse 1: 3.3 Nm\\

M$_{erforderlich}$ Drehachse 2: 5.4 Nm verteilt auf zwei Seiten

M$_{erforderlich}$ Drehachse 2 pro Seite: 2.7 Nm
