\section{Einleitung}

In der vorliegenden Dokumentation wird im Rahmen des Moduls \acrfull{pren2} ein autonomer Roboter nach dem Konzept aus \acrfull{pren1} umgesetzt und weiterentwickelt. Der zu realisierende Roboter muss in der Lage sein, die Aufgabenstellung am Stichtag autonom zu bewältigen. Diese Aufgabe umfasst das Erkennen verschiedener Piktogramme sowie das Erklimmen einer Treppe ohne fremde Hilfseinwirkung. Zusätzlich befinden sich auf der zu besteigenden Treppe verschiedene Hindernisse, welche geschickt um- oder übergangen werden müssen.

Dazu wird ein erster Prototyp konzeptgetreu gebaut und im Verlauf der Testphasen weiterentwickelt und verbessert. In diesem Vorgang spielen verschiedene Kompetenzen eine grosse Rolle, welche fortlaufend zusammenkommen, um das Ziel zu erreichen. Die Grenzen der drei beteiligten Studiengänge werden dementsprechend fortwährend aufgelöst und eine Vielzahl von Erfahrungen in allen Studienrichtungen kann erworben werden. Somit hat dieses Projekt das Potential für alle Beteiligten eine lehrreiche und unvergessliche Erfahrung zu werden.

In diesem Bericht wird nicht nur das Vorgehen erläutert und dokumentiert. Vom Start mit dem Konzept aus \acrshort{pren1} bis hin zur fertigen Lösung, mit welcher am Wettkampf teilgenommen wird, werden die Entwicklungen und die Überlegungen dargelegt und argumentativ begründet. Ausserdem werden sämtliche Tests dokumentiert. Vom ersten Prototypen mit nur einem Motor hin zu einem komplexen Gebilde aus mechanischen und elektrischen Komponenten, welche mit korrekter Software zu einem funktionierenden Roboter werden. Mit dem einzigen Ziel vor den Augen: Der Wettbewerbstag.

