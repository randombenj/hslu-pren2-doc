\section{Einleitung}

In der vorliegenden Dokumentation wird im Rahmen des Moduls \acrfull{pren2} ein autonomer Roboter nach dem Konzept aus \acrfull{pren1} umgesetzt und weiterentwickeltl. Der zu realisierende Roboter muss in der Lage sein, die Aufgabenstellung am Stichtag autonom zu bewältigen.

Dazu wird ein erster Prototyp konzeptgeträu gebaut und im Verlauf der Testphasen weiterentwickelt und verbessert. In diesem Vorgang spielen verschiedene Kompetenzen eine grosse Rolle, welche fortlaufend zusammenkommen um das Ziel zu erreichen. Die Grenzen der drei beteiligten Studiengänge werden dementsprechend fortwährend aufgelöst.

In diesem Bericht wird das Vorgehen erläutert und dokumentiert. Vom Start mit dem Konzept aus \acrfull{pren1} bis hin zur fertigen Lösung, mit welcher am Wettkampf teilgenommen wird, werden die Entwicklungen und die Überlegungen erläutert und argumentativ begründet. Ausserdem werden sämtliche Tests dokumentiert. Dabei finden zu Beginn Tests in den disziplinären Teams statt, welche mit voranschreitendem Entwicklungsstand fortlaufend interdisziplinärer werden.

