\section{Anforderungsliste}
\label{sec:anforderungsliste}
Die nachfolgende Tabelle \ref{tab:anforderungsliste} listet alle Anforderungen an den autonomen Baugerüstroboter auf. Hierbei wird zwischen Fest-, Mindest- und Wunschanforderungen unterschieden. Weiter werden die Anforderungen in Funktionelle und nicht Funktionelle unterteilt. Zudem wurden den Anforderungen eine Priorität und ein Komplexitäts-Level zugeteilt. Hierbei entspricht eine 1 dem höchsten Wert. Z.B. wurde der Anforderung autonomes Fahren die Priorität 1 als auch die Komplexität 1 zugeordnet. Das heisst, es hat höchste Priorität und höchste Komplexität.
\begin{items}
  \item {\bf F} = Festanforderung
  \item {\bf M} = Mindestanforderung
  \item {\bf W} = Wunschanforderung
\end{items}

\scriptsize
\begin{longtable}[]{@{}lllp{2cm}p{5cm}llp{2cm}@{}}
%\toprule
\textbf{Nr.} & \textbf{Kat.} & \textbf{Prio} & \textbf{Bezeichnung}
& \textbf{Erläuterung} & \textbf{Verantwortlich} & \textbf{Complexity} &
\textbf{Typ}\tabularnewline
%\midrule
\endhead

1.1 & F & 1 & Fahrzeug aus Eigenkonstruktion &
Fahrzeug muss selbst entwickelt
werden. Komponenten wie Sensoren, Servos, Kameras dürfen
verwendet werden. & Alle & 2 & nicht \hbox{funktionell}\tabularnewline

1.2 & F & 1 & Autonomes Fahren & Unter allen zu erwartenden Licht- und
Umgebungsbedingungen & I / E & 1 & funktionell\tabularnewline

1.3 & F & 2 & Start & Start via Startknopf & I / E & 3 &
funktionell\tabularnewline

1.4 & F & 3 & Witterungsbeständigkeit & \vtop{\hbox{\strut -
Windresistent}\hbox{\strut - Regenresistent}\hbox{\strut -
Sonnenresistent}\hbox{\strut - Schmutzresistent}} & E / M & 3 &
funktionell\tabularnewline

1.5 & M & 2 & Spritzwasserfest / Schmutzresistez & IP24 & E / M & 2 &
funktionell\tabularnewline

1.6 & M & 2 & Erschütterungsresistenz & Die Komponenten funktionieren
auch bei Erschütterung des Fahrzeugs. & Alle & 2 &
funktionell\tabularnewline

1.7 & M & 3 & Betriebstemperatur & Fahrzeug funktioniert bei
Aussentemperaturen von 5\textdegree C bis 35\textdegree C. & E & 3 &
funktionell\tabularnewline

1.8 & M & 1 & Umgebungserkennung & \vtop{\hbox{\strut - Markierung am
Boden des Startfeldes}\hbox{\strut - Piktogramm erkennen (20x20
cm)}\hbox{\strut - Treppe finden}\hbox{\strut - Hindernisse
erkennen}\hbox{\strut - Zielplattform erkennen}\hbox{\strut - Darf das
Geländer nicht berühren}} & I & 1 & funktionell\tabularnewline

1.9 & F & 1 & Hinderniss-Kollision & Hindernisse dürfen berührt, aber
nicht verschoben werden. & Alle & 2 & funktionell\tabularnewline

1.10 & F & 2 & Spielfeldboden & Fahrzeug kann folgendes Terrain
befahren: Holzboden, Baugerüst & M & 3 & funktionell\tabularnewline

1.11 & M & 1 & Stufen erklimmen & Das Fahrzeug kann 5 - 10 Stufen
(Tritthöhe: 20 cm, Tritttiefe: 30 cm) erklimmen. & E / M & 1 &
funktionell\tabularnewline

1.12 & F & 2 & Fahrzeugabmessung maximal 40x40cm & Fahrzeugabmessung darf
im Startzustand nicht grösser als 40x40 cm sein. Im Fahrzustand darf es
grösser sein. & Alle & 3 & nicht \hbox{funktionell}\tabularnewline

1.13 & F & 3 & Maximale Höhe & Maximal 100cm vom Boden entfernt & M & 3
& funktionell\tabularnewline

1.14 & F & 1 & Kommunikation mit Publikum &
Fahrzeug quittiert Auftrag sinnvoll und
erkennbar- Fahrzeug meldet, wenn gesuchtes Objekt
gefunden & I / E & 2 & funktionell\tabularnewline

1.15 & F & 3 & Fahrzeit & max. 4min & Alle & 2 &
funktionell\tabularnewline

1.16 & F & 3 & Lernfahrt & 2min vor Startbeginn für Vorbereitungen &
Alle & 3 & funktionell\tabularnewline

1.17 & M & 3 & Mobile Energiequelle & Das Fahrzeug muss fähig sein 2
Läufe zu absolvieren. (2x4min Laufzeit + 2x2min Vorbereitung = 12min) & E &
2 & funktionell\tabularnewline

1.18 & F & 2 & Gesamtbudget & max. 500 CHF & Alle & 2 & nicht
\hbox{funktionell}\tabularnewline

1.19 & F & 2 & Budget PREN1 & max. 200 CHF & Alle & 2 & nicht
\hbox{funktionell}\tabularnewline

1.20 & F & 2 & Keine Raupenfahrzeuge & Das Gerät darf fahren, schreiten,
rutschen, klammern, krabbeln, schweben oder auch tief fliegen. & Alle &
3 & funktionell\tabularnewline

1.21 & W & 3 & Gewicht & 3 kg (ungefährer Wert aus vergangenen
PREN-Projekten) & M & 2 & nicht \hbox{funktionell}\tabularnewline

1.22 & M & 2 & End-Position & Gerät muss automatisch stoppen und sich
vollständig auf dem Podest befinden. & I & 2 &
funktionell\tabularnewline

1.23 & W & 3 & Not-Aus-Taster & Das Gerät soll mit einem Not-Aus Taster
sofort gestoppt werden können. & E / I & 3 & funktionell\tabularnewline
\caption{Anforderungskatalog}
\label{tab:anforderungsliste}
%\bottomrule
\end{longtable}
\normalsize

\newpage