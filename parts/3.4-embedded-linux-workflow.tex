\newpage

\section{Embedded Linux Workflow}
In den nachfolgenden Abschnitten wird kurz vorgestellt, wie der Entwicklungsprozess mit dem Jetson Nano optimiert werden konnte. Der Entwicklungsprozess mit dem Raspberry Pi in den ersten zwei Sprints war im Grunde sehr ähnlich, die nachfolgende Beschreibung bezieht sich jedoch lediglich auf das Jetson Nano.

\subsection{Verwendete Versionen}
Es wird das bereits vorinstallierte Python 3.6 verwendet. In Kombination zu Python 3.6 wurde die aktuelle Version von pip 21.0 installiert. Um zu gewährleisten, dass die richtige Version von pip verwendet wird, empfiehlt es sich, pip über den python3 Befehl aufzurufen.

Möchte man beispielsweise numpy über pip installieren, kann folgender Befehl verwendet werden:
\begin{minted}{bash}
python3 -m pip install numpy
\end{minted}

Die Versionen der verwendeten Python Packages werden im File requirements.txt festgehalten. Mithilfe von pip kann man sämtliche Packages mit den entsprechenden Versionen aus der Datei installieren.

\begin{minted}{bash}
python3 -m pip -r install requirements.txt
\end{minted}

\subsection{SSH Verbindung}
Da das Jetson standardmässig kein kein WLAN-Modul verfügt, wird eine statische IP (192.168.137.153) für das Interface eth0 eingerichtet. Das Standard Gateway dieser statischen IP ist 192.168.137.1/255.255.255.0. Damit über ein Ethernet Kabel direkt auf das Jetson zugegriffen werden kann, muss dem entsprechenden Netzwerkinterface auf der Hostmaschine die IP des Default Gateways des Jetsons, also 192.168.137.1/255.255.255.0, zugewiesen werden. Somit kann über die IP 192.168.137.153 eine SSH Verbindung über Ethernet zum Jetson hergestellt werden. Damit das Jetson jedoch auch mit dem Internet verbunden ist, muss noch die aktuelle Internetverbindung mit dem entsprechenden Ethernet Interface geteilt werden. Dies geht unter Windows einfach über die Adapteroptionen in den Netzwerkeinstellungen.

\subsubsection{Terminal}
Um den Workflow noch zu beschleunigen, wurde ein entsprechender SSH Public Key auf dem Jetson platziert (\textasciitilde/.ssh/authorized\_keys). Dies ermöglicht eine Verbindung zum Jetson ohne die Eingabe eines Passwortes. Auf der Hostmaschine wurde in der Datei config im .ssh Order ein Host für das Jetson konfiguriert.

\begin{minted}{bash}
Host jetson
	HostName 192.168.137.153
	User jetson
	IdentityFile ~/.ssh/id_ed25519
\end{minted}

Dies ermöglicht es, aus einem Terminal mit dem nachfolgenden Befehl eine Verbindung ohne Passwort zum Jetson herzustellen.
\begin{minted}{bash}
ssh jetson
\end{minted}

\subsubsection{Putty}
Dasselbe wurde für Putty vorgenommen. Da Putty jedoch sein eigenes SSH Keyformat verwendet, musste der entsprechende Public Key welcher sich bereits auf dem Jetson befindet in ein Putty Public Key (.ppk) konvertiert werden. Dieser wurde ebenfalls in das File \textasciitilde/.ssh/authorized\_keys kopiert. 

\subsection{Git Workflow}
\subsubsection{Git remote repository}
In einer ersten Phase wurde auf dem Jetson ein remote repository eingerichtet, um vom Notebook direkt über ssh aufs Jetson pushen zu können. Hierzu waren drei schritte notwendig.

Erstellen des entprechenden Directory auf dem Jetson:
\begin{minted}{bash}
mkdir git/pren
\end{minted}


Hinzufügen des Remote Repositorys auf der Hostmaschine:
\begin{minted}{bash}
git remote add live ssh://jetson@192.168.137.153/~/git/pren
\end{minted}


Erster push zum Remote Repository:
\begin{minted}{bash}
git push live +master:refs/heads/master
\end{minted}


Somit war es möglich auf dem Notebook zu entwickeln und die änderungen immer direkt mit dem Befehl: git push live auf das Jetson zu pushen. Der grosse Nachteil hierbei ist, dass jegliche Änderungen am Code immer zuerst committed werden bevor diese auf dem Jetson getestet werden können. Die sorgt für eine sehr unübersichtliche Commithistory. Aus diesem Grund wurde ab Sprint 3 die nachfolgende Variante umgestiegen.

\subsubsection{VSCode Remote SSH}
VSCode beitet mit dem Extension Remote - SSH \cite{VSCode-Remote-SSH} an, die Filestruktur eines über SSH verbundenen Geräts im VSCode auf der Hostmaschine zu öffnen. Somit kann man ebenfalls auf dem Notebook entwickeln, greift jedoch direkt auf die Files auf dem Jetson zu und muss diese nicht zuerst pushen. Vorteil ist, dass man so die Commithistory sauber halten kann. Dies ist somit definitiv die bessere Variante für die Entwicklung. Die erste Variante eignet mehr, um bereits getesteten Code auf die Produktivumgebung zu deployen aber nicht für die aktive Entwicklung.