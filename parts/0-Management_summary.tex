\newpage
\section*{Management Summary}

Die vorliegende Projektarbeit des Moduls \acrfull{pren2} befasst sich mit der Umsetzung des Projektes aus \acrfull{pren1}, um praxisnahe Kompetenzen nicht nur im eigenen Studienbereich, sondern auch in der Zusammenarbeit mit anderen Studiengängen zu erlangen.

Die Arbeit wird teilweise unterteilt in die einzelnen Disziplinen, um die Nachvolziehbarkeit zu gewährleisten.

Das Ziel dieses Projektes ist die Realisierung der vorhergegangenen Planung eines vollautonomen Treppensteigroboter, welcher einen Gegenstand als Piktogramm im Startbereich erkennt. Anschliessend eine Treppe erklimmt, dabei verschiedenen Hindernissen ausweicht,  und im Zielbereich auf dem oberen Podest der Treppe aus verschiedenen Piktogrammen das richtige auswählt, berührt und diesen Fund signalisiert. Dieser Ablauf wird zwei mal durchgeführt an dem vorbestimmten Wettkampftag und pro Lauf stehen maximal vier Minuten zur Verfügung. Dabei ist eine externe Kommunikation und Raupensysteme zum Antrieb verboten.

Um das bestmögliche Ergebnis zu erreichen, setzt die Gruppe 5 bestehend aus zwei Informatik-,  zwei Maschinentechnik- und zwei Elektrotechnikstudenten auf möglichst viele Testphasen, bei welchen bereits früh im Entwicklungsprozess verschiedene Funktionen getestet werden. Dabei wird von der Konstruktionsplanung über die Komponentenauswahl alles aufgrund des bekannten Konzepts konstruiert wobei bessere Lösungen nicht ausgeschlossen werden und Verbesserungsmöglichkeiten in der Konstruktion häufig umgesetzt und getestet werden. Dadurch ist der abschliessende Lösungsansatz in der Grundform ähnlich aber nicht identisch zu jenem aus PREN 1.

Der finale Roboter mit dem Namen \glqq Hubert\grqq{} besteht grösstenteils aus einem Polylactid (PLA), wurde im 3D-Drucker hergestellt und beinhaltet einen Controller, welcher die Motoren ansteuert und die Sensoren ausliest. Um die nötige Rechenleistung zur Verfügung zu haben, wird ein Nvidia Jetson Nano verwendet, da dieses Board eine dedizierte Grafikkarte besitzt. Des weiteren werden für die Hubbewegung ein Getriebemotor und die Fortbewegung mit zwei abgewinkelten Getriebemotoren umgesetzt. Alle drei Motoren werden mit einem Hardware-PWM Extension-Board über die H-Brücken angesteuert. Zu den Sensoren zählt ein Ultraschallsensor zwei TOF-Sensoren welche über einen Multiplexer laufen. Diese unterstützen die Kamera bei der Orientierung, welche auch für die Piktogrammerkennung und die Pfadfindung zuständig ist.

In ersten Versuchen konnte die Treppe in 3.5 Minuten erklommen werden. Dabei ist die Fehlerquote relativ gering.


